\documentclass[12pt,a4paper,oneside]{article}

\usepackage[absolute,overlay]{textpos}
\usepackage{graphicx}
\usepackage{adjustbox}
\usepackage{chemfig}
\usepackage[version=4]{mhchem}
\usepackage{wrapfig}
\usepackage{multirow}
\adjustboxset*{center}
\usepackage{caption}
\usepackage{chemformula}
\usepackage{elements}

%dělení slov
\usepackage{ragged2e}
\let\raggedright=\RaggedRight
%konec dělení slov

\usepackage{fontspec}
\usepackage{unicode-math}

\usepackage{polyglossia}
\setdefaultlanguage{czech}

\def\uv#1{„#1“}

\begin{document}

\section{Termochemie}
\textbf{Vypočítejte reakční entalpii přeměny grafitu na diamant:}\\
\ce{C(gr) -> C(diam)}\\

jestliže znáte entalpie reakcí:\\

\begin{tabular}{clr@{,}l}
	A: & \ce{C(gr) + O2(g) -> CO2(g)} & $-$393 & 77 kJ.mol$^{-1}$ \\
	B: & \ce{C(diam) + O2(g) -> CO2(g)} & $-$395 & 65 kJ.mol$^{-1}$ \\
\end{tabular}
\\
Jelikož nás zajímá přeměna grafitu na diamant, vezmeme entalpii spalování grafitu a od ní odečteme entalpii spalování diamantu:\\
A$-$B\\
\ce{C(gr) + O2(g) + CO2(g) -> C(diam) + O2(g) + CO2(g)}
\\
Entalpii tedy vypočítáme:\\
$\Delta H_r\ =\ -393,77\ -(-395,65)\ =\ 1,88$ kJ.mol$^{-1}$
\\
\textbf{Entalpie přeměny grafitu na diamant bude 1,88 kJ.mol$^{-1}$}\\
\hrule

\textbf{Vypočítejte entalpii spalování acetylenu:}\\
\ce{C2H2(g) + \frac{5}{2} O2(g) -> 2 CO2(g) + H2O(l)}\\

jestliže znáte entalpie reakcí:\\

\begin{tabular}{clr@{,}l}
	A: & \ce{2 C(s) + H2(g) -> C2H2(g)} & 226 & 92 kJ.mol$^{-1}$ \\
	B: & \ce{2 C(s) + O2(g) -> CO2(g)} & $-$393 & 97 kJ.mol$^{-1}$ \\
	C: & \ce{H2(g) + \frac{1}{2} O2(g) -> H2O(l)} & $-$285 & 96 kJ.mol$^{-1}$ \\
\end{tabular}

Zadanou rovnici získáme následující kombinací známých reakcí:\\
$-$A+2B+C\\

Entalpii tedy vypočítáme:\\
$\Delta H_r\ =\ -226,92\ -\ 2.393,7\ - 285,96\ =\ -1300,82$ kJ.mol$^{-1}$

\textbf{Entalpie zadané reakce bude $-$1300,82 kJ.mol$^{-1}$}

\end{document}