\documentclass[hyperref=unicode]{beamer}

\usepackage[absolute,overlay]{textpos}
\usepackage{graphicx}
\usepackage{adjustbox}
\usepackage{chemfig}
\usepackage[version=4]{mhchem}
\usepackage{wrapfig}
\usepackage{multirow}
\adjustboxset*{center}
\usepackage{caption}
\usepackage{chemformula}
\usepackage{elements}

%dělení slov
\usepackage{ragged2e}
\let\raggedright=\RaggedRight
%konec dělení slov

\usepackage{fontspec}
\usepackage{unicode-math}

\usepackage{polyglossia}
\setdefaultlanguage{czech}

\def\uv#1{„#1“}

\mode<presentation>{\usetheme{Madrid}}
\DefineNamedColor{named}{pozadi}{RGB}{200,200,200}
\usecolortheme{crane}

\setbeamertemplate{footline}[frame number]

\addtobeamertemplate{frametitle}{
	\let\insertframetitle\insertsectionhead}{}
\addtobeamertemplate{frametitle}{
	\let\insertframesubtitle\insertsubsectionhead}{}

\makeatletter
\CheckCommand*\beamer@checkframetitle{\@ifnextchar\bgroup\beamer@inlineframetitle{}}
\renewcommand*\beamer@checkframetitle{\global\let\beamer@frametitle\relax\@ifnextchar\bgroup\beamer@inlineframetitle{}}
\makeatother
\setbeamercolor{section in toc}{fg=blue}
\setbeamertemplate{section in toc shaded}[default][100]

\title[Crisis] % (optional, only for long titles)
{Termodynamika}

\subtitle{Příklady výpočtů}

\date{}

\titlegraphic{}
\begin{document}

\frame{\titlepage}

\section{Termochemie}
\frame{
	\frametitle{}
	\textbf{Vypočítejte reakční entalpii přeměny grafitu na diamant:}\\
	\ce{C(gr) -> C(diam)}\\
	\vspace{5mm}
	jestliže znáte entalpie reakcí:\\
	\vspace{3mm}
	\begin{tabular}{clr@{,}l}
		A: & \ce{C(gr) + O2(g) -> CO2(g)} & $-$393 & 77 kJ.mol$^{-1}$ \\
		B: & \ce{C(diam) + O2(g) -> CO2(g)} & $-$395 & 65 kJ.mol$^{-1}$ \\
	\end{tabular}
	\\
	\vspace{3mm}
	Jelikož nás zajímá přeměna grafitu na diamant, vezmeme entalpii spalování grafitu a od ní odečteme entalpii spalování diamantu:\\
	A$-$B\\
	\ce{C(gr) + O2(g) + CO2(g) -> C(diam) + O2(g) + CO2(g)}
	\\
	\vspace{3mm}
	Entalpii tedy vypočítáme:\\
	$\Delta H_r\ =\ -393,77\ -(-395,65)\ =\ 1,88$ kJ.mol$^{-1}$
	\\
	\vspace{5mm}
	\textbf{Entalpie přeměny grafitu na diamant bude 1,88 kJ.mol$^{-1}$}
	\vfill
}

\frame{
	\frametitle{}
	\textbf{Vypočítejte entalpii spalování acetylenu:}\\
	\ce{C2H2(g) + \frac{5}{2} O2(g) -> 2 CO2(g) + H2O(l)}\\
	\vspace{5mm}
	jestliže znáte entalpie reakcí:\\
	\vspace{3mm}
	\begin{tabular}{clr@{,}l}
		A: & \ce{2 C(s) + H2(g) -> C2H2(g)} & 226 & 92 kJ.mol$^{-1}$ \\
		B: & \ce{2 C(s) + O2(g) -> CO2(g)} & $-$393 & 97 kJ.mol$^{-1}$ \\
		C: & \ce{H2(g) + \frac{1}{2} O2(g) -> H2O(l)} & $-$285 & 96 kJ.mol$^{-1}$ \\
	\end{tabular}
	\\
	\vspace{3mm}
	Zadanou rovnici získáme následující kombinací známých reakcí:\\
	$-$A+2B+C\\
	Entalpii tedy vypočítáme:\\
	$\Delta H_r\ =\ -226,92\ -\ 2.393,7\ - 285,96\ =\ -1300,82$ kJ.mol$^{-1}$
	\\
	\vspace{5mm}
	\textbf{Entalpie zadané reakce bude $-$1300,82 kJ.mol$^{-1}$}
	\vfill
}

\end{document}