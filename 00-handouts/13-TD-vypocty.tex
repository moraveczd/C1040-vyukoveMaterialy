
\section{Termodynamika}

\subsection{Zákony termodynamiky}

\textbf{Termodynamika} je obor fyziky, který se zabývá procesy a vlastnostmi látek a polí spojených s teplem a tepelnými jevy; je součástí termiky. Vychází přitom z obecných principů přeměny energie, které jsou popsány čtyřmi termodynamickými zákony (z historických důvodů číslovány nultý až třetí):\\

\textbf{Nultý zákon TD}

Jsou-li dvě a více těles v termodynamické rovnováze s tělesem dalším, pak jsou všechna tato tělesa v rovnováze.

\textbf{První zákon TD}

Celkové množství energie (všech druhů) izolované soustavy zůstává zachováno.

Nelze sestrojit stroj, který by trvale dodával mechanickou energii, aniž by spotřeboval odpovídající množství energie jiného druhu.

\textbf{Druhý zákon TD}

Teplo nemůže při styku dvou těles různých teplot samovolně přecházet z tělesa chladnějšího na těleso teplejší.

Nelze sestrojit periodicky pracující tepelný stroj, který by trvale konal práci pouze tím, že by ochlazoval jedno těleso, a k žádné další změně v okolí by nedocházelo.

\textbf{Třetí zákon TD}

Při absolutní nulové teplotě je entropie čisté látky pevného nebo kapalného skupenství rovna nule.

\newpage

\subsection{Termochemie}
\emph{Vypočítejte reakční entalpii přeměny grafitu na diamant:}

\ce{C(gr) -> C(diam)}\\
jestliže znáte entalpie reakcí:\\

\begin{tabular}{clr@{,}l}
	A: & \ce{C(gr) + O2(g) -> CO2(g)} & $-$393 & 77 kJ.mol$^{-1}$ \\
	B: & \ce{C(diam) + O2(g) -> CO2(g)} & $-$395 & 65 kJ.mol$^{-1}$ \\
\end{tabular}

Jelikož nás zajímá přeměna grafitu na diamant, vezmeme entalpii spalování grafitu a od ní odečteme entalpii spalování diamantu: \textbf{A$-$B}\\

\ce{C(gr) + O2(g) + CO2(g) -> C(diam) + O2(g) + CO2(g)}

Entalpii tedy vypočítáme:\\
$\Delta H_r\ =\ -393,77\ -(-395,65)\ =\ 1,88$ kJ.mol$^{-1}$
\\
\textbf{Entalpie přeměny grafitu na diamant bude 1,88 kJ.mol$^{-1}$}\\
\hrule
\vspace{2mm}
\emph{Vypočítejte entalpii spalování acetylenu:}

\ce{C2H2(g) + \frac{5}{2} O2(g) -> 2 CO2(g) + H2O(l)}\\
jestliže znáte entalpie reakcí:\\

\begin{tabular}{clr@{,}l}
	A: & \ce{2 C(s) + H2(g) -> C2H2(g)} & 226 & 92 kJ.mol$^{-1}$ \\
	B: & \ce{2 C(s) + O2(g) -> CO2(g)} & $-$393 & 97 kJ.mol$^{-1}$ \\
	C: & \ce{H2(g) + \frac{1}{2} O2(g) -> H2O(l)} & $-$285 & 96 kJ.mol$^{-1}$ \\
\end{tabular}

Zadanou rovnici získáme následující kombinací známých reakcí:\\
$-$A+2B+C\\

Entalpii tedy vypočítáme:\\

$\Delta H_r\ =\ -226,92\ -\ 2.393,7\ - 285,96\ =\ -1300,82$ kJ.mol$^{-1}$\\
\textbf{Entalpie zadané reakce bude $-$1300,82 kJ.mol$^{-1}$}
\newpage

\subsection{Hessův zákon}