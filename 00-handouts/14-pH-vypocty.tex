\section{pH}

\subsection{Vzorce}

\begin{tabular}{ll}
	Silná kyselina & pH = $-\log$ c \\
	& \\
	Silná zásada & pH = 14 + $\log$ c \\
	& \\
	Slabá kyselina & pH = $\frac{1}{2}\textrm{p}K_A-\frac{1}{2}\log$ c \\
	& \\
	Slabá zásada & pH = $14\ +\ \frac{1}{2}\log\textrm{c} - \frac{1}{2}\textrm{p}K_B$ \\
	& \\
	Sůl slabé k a silné z & pH = $7\ +\ \frac{1}{2}\log\textrm{c} + \frac{1}{2}\textrm{p}K_A$ \\
	& \\
	Sůl silné k a slabé z & pH = $7\ -\ \frac{1}{2}\log\textrm{c} - \frac{1}{2}\textrm{p}K_B$ \\
	& \\
	Sůl slabé k a slabé z & pH = $7\ +\ \frac{1}{2}\textrm{p}K_A - \frac{1}{2}\textrm{p}K_B$ \\
	& \\
	Pufr -- kyselina & pH = $\textrm{p}K_A + \log \frac{[A^-]}{[HA]}$ \\
	& \\
	Pufr -- zásada & pH = 14 - $\textrm{p}K_B - \log \frac{[B^+]}{[BOH]}$ \\
\end{tabular}

\subsection{Iontový součin vody}
\ce{H2O + H2O <=> H3O+ + OH-}

$\textrm{K = }\frac{[\ce{H3O+}][\ce{OH-}]}{[\ce{H2O}]^2}$

K$_w$ = \ce{[H3O+][OH-]} = 10$^{-14}$

pK$_w$ = pH + pOH = 14

\pagebreak

\subsection{Silné kyseliny a zásady}
\textit{Vypočítej pH kyseliny chlorovodíkové o koncentraci 0,3 M.}

\ce{HCl -> H+ + Cl-}

pH = -log c = -log 0,3 = 0,52

\textit{Vypočítej pH kyseliny sírové o koncentraci 0,3 M.}

\ce{H2SO4 -> 2 H+ + SO$_4^{2-}$}

pH = -log c = -log (2\times 0,3) = 0,22

\textit{Vypočítej pH hydroxidu sodného o koncentraci 0,3 M.}

\ce{NaOH -> Na+ + OH-}

pOH = -log c = -log 0,3 = 0,52

pH = 14 - pH = 14 - 0,52 = 13,48

\newpage

\subsection{Slabé kyseliny a zásady}
\emph{Jaké je pH 0,2 M kyseliny octové, p$K_a$ = 4,76?}

\ce{CH3COOH <=> CH3COO^- + H^+}

$K_a = 10^{-\textrm{pK}_a} = 10^{-4.76} = 0,000017$

$K_a = \frac{[\ce{CH3COO^-}][\ce{H^+}]}{[\ce{CH3COOH}]} = \frac{x.x}{0,2-x}$

Dosadíme za $K_a$ a upravíme získaný výraz, čímž dostaneme kvadratickou rovnici:

$x^2 + 0,000017x - 0,0000034 = 0$

Kvadratickou rovnici vyřešíme pomocí diskriminantu:

$x_{1,2} = \frac{-b \pm \sqrt{D}}{2a} = \frac{-b \pm \sqrt{b^2 - 4ac}}{2a} = \frac{-0,000017 \pm \sqrt{0,000017^2 - 4.1.(-0,0000034)}}{2.1}$

Ze dvou vypočítaných kořenů zvolíme ten kladný, koncentrace totiž nemůže být záporná.

$x = 0,001835$

$\textrm{pH} = -\log[H^+] = -\log0,01835 = 2,736$

\textbf{Zjednodušený výpočet}
$K_a = \frac{[\ce{CH3COO^-}][\ce{H^+}]}{[\ce{CH3COOH}]} = \frac{x.x}{0,2}$

Dosadíme za $K_a$ a upravíme získaný výraz, čímž dostaneme kvadratickou rovnici:

$x^2 - 0,0000034 = 0\\
x = \pm\sqrt{0,0000034}\\
x = 0,001844$

pH = -log 0,001844 = 2,734

\subsection{Soli}
\subsubsection{Sůl silné kyseliny a silné zásady}

\ce{NaCl -> Na+ + Cl-}

Při disociaci nedochází ke vzniku \ce{H+}, ani \ce{OH-} iontů, hodnota pH tedy není ovlivněna.

\subsection{Pufry}


\clearpage