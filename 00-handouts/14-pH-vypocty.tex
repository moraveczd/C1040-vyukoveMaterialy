\section{pH}

\subsection{Aktivita}

Popisuje reálné chování roztoku. Na rozdíl od ideálního roztoku, se v reálném roztoku částice navzájem ovlivňují. Aktivita jakékoliv čisté látky v kondenzovaném stavu (kapalina nebo pevná látka) je jednotková. Aktivita plynu závisí na jeho parciálním tlaku, obvykle se označuje jako \textbf{fugacita}.

$\mu_i = \mu_i^0 + \textrm{RT} \ln a_i$\\
$\mu_i$ - chemický potenciál, $\mu_i^0$ - standardní chemický potenciál

Ativitu lze vyjádřit jako součin molární koncentrace a aktivitního koeficientu:\\
a $ = \gamma $c

Aktivitní koeficient je úměrný náboji iontů v roztoku a iontové síle roztoku

\subsubsection{Iontová síla roztoku}
$\log \gamma = -0,509\textrm{z}^2\sqrt{\textrm{I}}$\\
I - iontová síla roztoku - popisuje množství iontů v roztoku\\
$I = \frac{1}{2}\sum\limits_{i=0}^n \textrm{c}_i \textrm{z}^2_i$\\
c$_i$ -- molalita; z$_i$ -- náboj; 0,509 -- konstanta pro vodné roztoky při 25~$^\circ$C

\textbf{Střední aktivitní koeficienty ve vodných roztocích při 25 $^\circ$C}\footnote{VOHLÍDAL, Jiří. Chemické tabulky. Praha: SNTL, 1982.}\\
\begin{tabular}{|l||r@{,}l|r@{,}l|r@{,}l|r@{,}l|}
\hline
$c_m [mol.kg^{-1}]$ & 0 & 1 & 1 & 0 & 4 & 0 & 10 & 0 \\\hline
HCl & 0 & 796 & 0 & 809 & 1 & 762 & 10 & 44 \\\hline
NaOH & 0 & 766 & 0 & 678 & 0 & 903 & 3 & 52 \\\hline
KOH & 0 & 798 & 0 & 756 & 1 & 352 & 6 & 22 \\\hline
\ce{H_2SO_4} & 0 & 265 & 0 & 130 & 0 & 171 & 0 & 553 \\\hline
\ce{AgNO_3} & 0 & 734 & 0 & 429 & 0 & 210 & \multicolumn{2}{|c|}{} \\\hline
\ce{Ca(NO_3)_2} & 0 & 48 & 0 & 35 & 0 & 42 & \multicolumn{2}{|c|}{} \\\hline
\end{tabular}


\subsection{Vzorce}

\begin{tabular}{ll}
	Silná kyselina & pH = $-\log$ c \\
	& \\
	Silná zásada & pH = 14 + $\log$ c \\
	& \\
	Slabá kyselina & pH = $\frac{1}{2}\textrm{p}K_A-\frac{1}{2}\log$ c \\
	& \\
	Slabá zásada & pH = $14\ +\ \frac{1}{2}\log\textrm{c} - \frac{1}{2}\textrm{p}K_B$ \\
	& \\
	Sůl slabé k a silné z & pH = $7\ +\ \frac{1}{2}\log\textrm{c} + \frac{1}{2}\textrm{p}K_A$ \\
	& \\
	Sůl silné k a slabé z & pH = $7\ -\ \frac{1}{2}\log\textrm{c} - \frac{1}{2}\textrm{p}K_B$ \\
	& \\
	Sůl slabé k a slabé z & pH = $7\ +\ \frac{1}{2}\textrm{p}K_A - \frac{1}{2}\textrm{p}K_B$ \\
	& \\
	Pufr -- kyselina & pH = $\textrm{p}K_A + \log \frac{[A^-]}{[HA]}$ \\
	& \\
	Pufr -- zásada & pH = 14 - $\textrm{p}K_B - \log \frac{[B^+]}{[BOH]}$ \\
\end{tabular}

\subsection{Iontový součin vody}
\ce{H2O + H2O <=> H3O+ + OH-}

$\textrm{K = }\frac{[\ce{H3O+}][\ce{OH-}]}{[\ce{H2O}]^2}$

K$_w$ = \ce{[H3O+][OH-]} = 10$^{-14}$

pK$_w$ = pH + pOH = 14

\pagebreak

\subsection{Silné kyseliny a zásady}
\textit{Vypočítej pH kyseliny chlorovodíkové o koncentraci 0,3 M.}

\ce{HCl -> H+ + Cl-}
pH = -log c = -log 0,3 = 0,52\\
\hrule
\textit{Vypočítej pH kyseliny sírové o koncentraci 0,3 M.}
\hrule
\ce{H2SO4 -> 2 H+ + SO$_4^{2-}$}\\
pH = -log c = -log (2\times 0,3) = 0,22

\textit{Vypočítej pH hydroxidu sodného o koncentraci 0,3 M.}

\ce{NaOH -> Na+ + OH-}

pOH = -log c = -log 0,3 = 0,52\\
pH = 14 - pH = 14 - 0,52 = 13,48
\hrule
\textit{Vypočítej pH kyseliny chlorovodíkové o koncentraci 1$\times$10$^{-8}$ M.}

Chybný výpočet: pH = -log 1.10$^{-8}$ = 8

Je nutné uvažovat iontový součin vody, pro správný výpočet musíme uvažovat tři podmínky:

\begin{enumerate}
	\item \ce{[Cl^-]} = c
	\item \ce{[H+]} = [\ce{OH-}] + [\ce{Cl-}]
	\item K$_w$ = \ce{[H+]}[\ce{OH-}]
\end{enumerate}

Tím získáme kvadratickou rovnici:\\
$x^2 - cx - K_w = 0\\
\textrm{c} = 1,051\times10^{-7}\\
\textrm{pH} = \textbf{6,978}$


\newpage

\subsection{Slabé kyseliny a zásady}
\emph{Jaké je pH 0,2 M kyseliny octové, p$K_a$ = 4,76?}

\ce{CH3COOH <=> CH3COO^- + H^+}

$K_a = 10^{-\textrm{pK}_a} = 10^{-4.76} = 0,000017$

$K_a = \frac{[\ce{CH3COO^-}][\ce{H^+}]}{[\ce{CH3COOH}]} = \frac{x.x}{0,2-x}$

Dosadíme za $K_a$ a upravíme získaný výraz, čímž dostaneme kvadratickou rovnici:

$x^2 + 0,000017x - 0,0000034 = 0$

Kvadratickou rovnici vyřešíme pomocí diskriminantu:

$x_{1,2} = \frac{-b \pm \sqrt{D}}{2a} = \frac{-b \pm \sqrt{b^2 - 4ac}}{2a} = \frac{-0,000017 \pm \sqrt{0,000017^2 - 4.1.(-0,0000034)}}{2.1}$

Ze dvou vypočítaných kořenů zvolíme ten kladný, koncentrace nemůže být záporná.

$x = 0,001835$

$\textrm{pH} = -\log[H^+] = -\log0,01835 = 2,736$

\textbf{Zjednodušený výpočet}

$K_a = \frac{[\ce{CH3COO^-}][\ce{H^+}]}{[\ce{CH3COOH}]} = \frac{x.x}{0,2}$

Dosadíme za $K_a$ a upravíme získaný výraz, čímž dostaneme kvadratickou rovnici:

$x^2 - 0,0000034 = 0\\
x = \pm\sqrt{0,0000034}\\
x = 0,001844$

pH = -log 0,001844 = 2,734

\textbf{Vzorec pro výpočet pH:}

pH = $\frac{1}{2}\textrm{p}K_A-\frac{1}{2}\log$ c = $\frac{1}{2}\times4,76 -\frac{1}{2}\log$ 0,2 = 2,73

\newpage

\textit{Jaké je pH 0,25 M roztoku amoniaku}

\ce{NH3 + H2O -> NH$^+_4$ + OH-}

pK$_B$ = 4,76\\
K$_B$ = 1,74$\times$10$^{-5}$

K$_B$ = $\frac{[\ce{NH$^+_4$}][\ce{OH-}]}{[\ce{NH3}]}\ =\ \frac{\ce{x^2}}{\ce{c}}\\
\textrm{x = }\sqrt{\textrm{K}_B\times c}\ =\ \sqrt{1,74\times10^{-5}\times 0,25}\ =\ 2,09\times10^{-3}$\\
pOH = -log $2,09\times10^{-3}$ = 2,68\\
pH = 14 - 2,68 = \textbf{11,32}

\textbf{Vzorec:}\\
pH = $14\ +\ \frac{1}{2}\log\textrm{c} - \frac{1}{2}\textrm{p}K_B$ = 14 + 0,5 log 0,25 - 0,5$\times$4,76 = 11,32

\subsection{Soli}

\subsubsection{Sůl silné kyseliny a silné zásady}

\ce{NaCl -> Na+ + Cl-}

Při disociaci nedochází ke vzniku \ce{H+}, ani \ce{OH-} iontů, hodnota pH tedy není ovlivněna.

\subsubsection{Sůl silné kyseliny a slabé zásady}
\ce{NH4NO3 + H2O -> NH4^+ + NO3^- + NH3 + H^+}\\
$\textrm{pH} = 7 - \frac{1}{2}(\textrm{pK}_b + \log c)$ \\

\subsubsection{Sůl slabé kyseliny a silné zásady}
\ce{NaF + H2O -> Na^+ + F^- + HF + OH^-}\\
$\textrm{pH} = 7 + \frac{1}{2}(\textrm{pK}_a + \log c)$ \\

\subsubsection{Sůl slabé kyseliny a slabé zásady}
\ce{NH4F + H2O -> NH4^+ + F^- + NH3 + H^+ + HF + OH^-}\\
$\textrm{pH} = 7 + \frac{1}{2}(\textrm{pK}_a - \textrm{pK}_b)$\\
$\textrm{pK}_a (\ce{HF}) = 3,17$\\
$\textrm{pK}_b (\ce{NH3}) = 4,75$\\
$\textrm{pH} = 7 + \frac{1}{2}(3,17 - 4,75) = 6,21$\\

\newpage
\subsection{Pufry}

Jde o směs slabé kyseliny a její soli nebo slabé zásady a její soli. Příkladem je např. acetátový pufr - směs kyseliny octové a octanu sodného.

Rovnováhy v pufru lze popsat rovnicemi:\\
\ce{CH3COOH + H2O <-> CH3COO^- + H3O^+}\\
\ce{CH3COONa + H2O <-> CH3COOH + Na^+ + OH^-}\\
Přídavkem kyseliny vzniknou molekuly kyseliny octové, přídavkem zásady ionty octanu. pH roztoku se nezmění.\\
$\textrm{pH} = \textrm{pK}_a + \log \frac{[A^-]}{[HA]}$\\
$\textrm{pH} = 14 - \textrm{pK}_b + \log \frac{[B]}{[BH^+]}$


\begin{tabular}{|c|c|c|}
	\hline
	\textbf{Pufr} & \textbf{Složení} & \textbf{Rozsah pH} \\\hline
	Acetátový & \ce{CH3COOH}/\ce{CH3COONa} & 3,8 - 5,8 \\\hline
	Fosfátový & \ce{NaH2PO4}/\ce{Na2HPO4} & 6,2 - 8,2 \\\hline
	Borátový & \ce{H3BO3}/\ce{Na2B4O7} & 8,25 - 10,25 \\\hline
\end{tabular}


\clearpage