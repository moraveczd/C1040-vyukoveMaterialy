
\section{Platné číslice}

\textbf{Platné číslice} jsou číslice odečtené z měřící stupnice přístroje, vč. posledního odhadnutého místa.\footnote{\href{https://is.muni.cz/do/sci/UChem/um/laboratorni_technika/pages/uvod.html}{Chyby měření}}

Nuly mezi desetinnou čárkou a první nenulovou číslicí nejsou platné číslice:

\textbf{25,23} cm = \textbf{2523}00 $\mu$m = \textbf{2,523}$\times$10$^5$ $\mu$m

Platné jsou vždy čtyři číslice, vyznačený tučnou sazbou.

Při \textbf{sčítání a odečítání} má výsledek tolik \textit{desetinných míst} jako číslo s nejmenším počtem desetinných míst:

2,5 cm + 5,236 cm = 7,7 cm\\
2,5 cm + 3,3 $\mu$m = 2,5 cm

Při \textbf{násobení a dělení} má výsledek tolik \textit{platných číslic} jako číslo s nejmenším počtem platných číslic (platné číslice jsou vyznačeny tučně):

n = c . V = 0,0\textbf{50} . 0,0\textbf{1235} = 0,000\textbf{61}75 mol = 6,2$\times10^{-4}$ mol

\textbf{Při výpočtech se vždy zaokrouhluje až poslední výsledek. Zaokrouhlování mezivýpočtů zvyšuje chybu výpočtu.}