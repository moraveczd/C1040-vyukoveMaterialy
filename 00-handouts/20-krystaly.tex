\section{Krystaly}

V elementární buňce rozlišujeme čtyři typy poloh:
\begin{enumerate}
	\item Poloha uvnitř buňky, atom patří celý do jediné buňky
	\item Poloha ve středu stěny, atom je sdílen dvěma buňkami. V konkrétní buňce je umístěna polovina atomu.
	\item Poloha ve středu hrany, atom je sdílen čtyřmi buňkami. V konkrétní buňce je umístěna čtvrtina atomu.
	\item Poloha ve vrcholu buňky, atom je sdílen osmi buňkami. V konkrétní buňce je umístěna osmina atomu.
\end{enumerate}

\begin{figure}[h]
	\adjincludegraphics[width=.3\textwidth]{img/08-CsCl.png}
	\caption[Krystalová struktura chloridu cesného.]{Krystalová struktura chloridu cesného.\footnotemark}
\end{figure}
\footnotetext{Zdroj: \href{https://commons.wikimedia.org/wiki/File:Caesium-chloride-unit-cell-3D-balls.png}{Benjah-bmm27/Commons}}

Např. chlorid cesný obsahuje cesný ion ve středu kubické buňky a osm chloridových aniontů v jejích vrcholech. Cesný kation patři do krystalové buňky celý a každý chlorid tam spadá $\frac{1}{8}$, tzn. vzorec je CsCl$_{8\times\frac{1}{8}}$ = CsCl.

\begin{figure}[h]
	\adjincludegraphics[width=.3\textwidth]{img/08-TiO2.png}
	\caption[Krystalová struktura chloridu cesného.]{Krystalová struktura oxidu titaničitého.\footnotemark}
\end{figure}
\footnotetext{Zdroj: \href{https://commons.wikimedia.org/wiki/File:Kristallstruktur_Titandioxid.png}{Orci/Commons}}

\begin{enumerate}
	\item Ti: šedé, 8$\times\frac{1}{8}$ + 1 = 2
	\item O: červené, 2$\times$1 + 4$\times\frac{1}{2}$ = 4
\end{enumerate}

\clearpage