\documentclass[hyperref=unicode]{beamer}

\usepackage[absolute,overlay]{textpos}
\usepackage{graphicx}
\usepackage{adjustbox}
\usepackage{mhchem}
\usepackage{wrapfig}
\usepackage{multirow}
\adjustboxset*{center}
\usepackage[utf8]{inputenc}
\usepackage{caption}

%dělení slov
\usepackage{ragged2e}
\let\raggedright=\RaggedRight
%konec dělení slov

\usepackage{fontspec}
\usepackage{unicode-math}

\usepackage{polyglossia}
\setdefaultlanguage{czech}

\def\uv#1{„#1“}

\mode<presentation>{\usetheme{Madrid}}
\DefineNamedColor{named}{pozadi}{RGB}{200,200,200}
\usecolortheme{crane}

\setbeamertemplate{footline}[frame number]

\addtobeamertemplate{frametitle}{
	\let\insertframetitle\insertsectionhead}{}
\addtobeamertemplate{frametitle}{
	\let\insertframesubtitle\insertsubsectionhead}{}

\makeatletter
\CheckCommand*\beamer@checkframetitle{\@ifnextchar\bgroup\beamer@inlineframetitle{}}
\renewcommand*\beamer@checkframetitle{\global\let\beamer@frametitle\relax\@ifnextchar\bgroup\beamer@inlineframetitle{}}
\makeatother
\setbeamercolor{section in toc}{fg=blue}
\setbeamertemplate{section in toc shaded}[default][100]

\usepackage{tikz}
\usetikzlibrary{positioning}
\usetikzlibrary{arrows}
\usetikzlibrary{shapes.multipart}

\title[Crisis]
{Atomové jádro, elektronový obal}
%\subtitle{\url{http://z-moravec.net/chemie/zaklady-chemie/}}
\subtitle{}
\date{}

\begin{document}
\frame{\titlepage}

\section{Atomové jádro}
\frame{
	\frametitle{}
	\begin{itemize}
	\item Atomové jádro je tvořeno protony a neutrony
	\item Prvek je látka skládající se z atomů se stejným počtem protonů
	\item Nuklid je systém tvořený prvky se stejným počtem neutronů
	\item Izotopy jsou atomy prvku s různým počtem neutronů
	\begin{itemize}
		\item $^{12}_{\ 6}C$, $^{13}_{\ 6}C$, $^{14}_{\ 7}N$, $^{15}_{\ 7}N$
	\end{itemize}
	\item $^A_ZX$
	\begin{itemize}
	\item A - nukleonové číslo - počet nukleonů (protonů a neutronů) v jádře
	\item Z - protonové číslo - počet protonů v jádře
	\end{itemize}
	\item \textbf{Relativní atomová hmotnost} je dána hmotnostním poměrem atomových hmotností jednotlivých izotopů prvku.
	\item Chlor: $^{35}$Cl (75,529 \%), $^{37}$Cl (24,471 \%)
	\item $Ar(Cl) = w(^{35}Cl)\cdot A(^{35}Cl) +w(^{37}Cl)\cdot A(^{37}Cl) = 0,75529\cdot 34,97 + 0,24471\cdot 36,97 = 35,45$
	\end{itemize}
}

\section{Stabilita atomových jader}
\frame{
	\frametitle{}
	\begin{itemize}
	\item Na stabilitu má vliv velikost vazebné energie jádra a poměr mezi počtem protonů a neutronů. U lehkých jader je poměr zhruba 1:1, se vzrůstajícím protonovým číslem dochází ke zvyšování přebytku neutronů
	\item Vazebná energie je energie, která se uvolní při vzniku jádra z volných nukleonů
	\item Nejvíce stabilních jader má protonové i neutronové číslo sudé, např.~\ce{^{12}_6C}, \ce{^{16}_8O}, ...
	\item Naopak kombinace lichého protonového a neutronového čísla je u stabilních jader vzácná, známe pouze čtyři: \ce{^1_1H}, \ce{^6_3Li}, \ce{^{10}_5B} a \ce{^{14}_7N}
	\end{itemize}
}

\section{Radioaktivní rozpady}
\frame{
	\frametitle{}
	\begin{itemize}
	\item Pokud je v jádru nadbytek neutronů nebo protonů, jádro se přemění na stabilnější.
	\begin{itemize}
	\item $\alpha$ rozpad - rozpad charakteristický pro těžší jádra, dojde k uvolnění $\alpha$-částice (jádro \ce{^4_2He^{2+}}), vzniklé jádro má protonové číslo menší o 2 a~nukleonové o 4
	\item \ce{^{226}_{88}Ra -> ^{222}_{86}Rn + ^4_2He}
	\item V případě nadbytku neutronů může dojít k rozpadu neutronu na proton a elektron, během přeměny se uvolňuje částice $\beta^-$ ($^0_{-1}e^-$)
	\item \ce{^{32}_{15}P -> ^{32}_{16}S + ^0_{-1}e}
	\item V případě nadbytku protonů může dojít k rozpadu protonu na neutron a pozitron, během přeměny se uvolňuje částice $\beta^+$ ($^0_{+1}e^+$)
	\item \ce{^{11}_{6}C -> ^{11}_{5}B + ^0_{+1}e}
	\item Nadbytek protonů v jádře může být kompenzován i pomocí \textit{elektronového záchytu}, kdy proton pohltí elektron a vznikne neutron
	\item \ce{^{7}_{4}Be + ^0_{-1}e -> ^{7}_{3}Li}
	\end{itemize}
	\end{itemize}
}

\section{Jaderné reakce}
\frame{
	\frametitle{}
	\begin{itemize}
	\item \textbf{Poločas rozpadu} - doba, za kterou dojde k rozpadu poloviny jader v systému
	\item Pravděpodobnostní veličina
	\item Charakteristika nestabilních jader, pohybuje se od zlomků sekund až po milióny let
	\item $\frac{dN}{dt} = -\lambda N$
	\item $N(t) = N_0 e^{-\lambda t}$
	\item \scalebox{1.5}{$t_{\frac{1}{2}} = \frac{\ln 2}{\lambda} = \tau \ln 2$}
	\begin{itemize}
	\item $N$ - počet částic
	\item $N_0$ - počet částic na počátku
	\item $\lambda$ - rozpadová konstanta
	\item $\tau$ - doba života jádra - $\tau = \frac{1}{\lambda}$
	\end{itemize}
	\end{itemize}
}

\section{Elektronový obal}
\frame{
	\frametitle{}
	\begin{itemize}
	\item Elektrony vázané k atomovému jádru
	\item Elektronový obal tvoří asi 0,01 \% hmotnosti atomu, ale tvoří většinu jeho objemu
	\item Poloměr elektronového obalu je řádově $10^{-10}$ m
	\item Elektrony vykazují dualitu chování, v důsledku Heisenbergova principu neurčitosti nelze přesně určit polohu elektrou v atomu, proto popisujeme pouze pravděpodobnost výskytu elektronu
	\item Počet elektronů v obalu atomu (elektroneutrální částice) je shodný s~počtem protonů v jádře
	\item Elektrony se v obalu pohybují v prostoru vymezeném řešením Schrödingerovy rovnice, tento prostor označujeme jako \textbf{atomový orbital}
	\item \textbf{Valenční elektrony} - elektrony v poslední zaplněné slupce obalu, účastní se chemických dějů
	\end{itemize}
}

\frame{
	\frametitle{}
	\begin{itemize}
	\item Elektron v atomu můžeme popsat čtyřmi kvantovými čísly
	\begin{itemize}
	\item Hlavní kvantové číslo (n) - popisuje příslušnost orbitalu do elektronové slupky -- velikost orbitalu. Nabývá hodnot větších než 0.
	\item Vedlejší kvantové číslo (l) - popisuje tvar orbitalu. Často se používá označení pomocí písmen: s, p, d, f, g, h, ... Nabývá hodnot v~intervalu $<0, n-1>$.
	\item Magnetické kvantové číslo (m) - popisuje prostorovou orientaci orbitalu. Nabývá hodnot v intervalu $<-l; l>$.
	\item Spinové kvantové číslo (s) - nepopisuje orbital, ale spin elektronu v~orbitalu. Nabývá hodnot $\pm$\textonehalf.
	\end{itemize}
	\item \textbf{Pauliho princip výlučnosti} - v atomu nemohou existovat dva elektrony, které by měly shodná všechna čtyři kvantová čísla, musí se lišit alespoň spinem, tzn. že do jednoho atomového orbitalu se vejdou maximálně dva elektrony.
	\item \textbf{Výstavbový (Aufbau) princip} - elektrony zaplňují orbitaly od energeticky nejnižších. První jsou zaplňovány volné orbitaly s~nejnižším součtem n+l.
	\end{itemize}
}

\section{Elektronová konfigurace}
\frame{
	\frametitle{}
	\begin{itemize}
	\item Popisuje zaplnění atomových orbitalů elektrony
	\item Orbitaly jsou zaplňovány v pořadí: 1s, 2s, 2p, 3s, 3p, 4s, 3d, 4p, 5s, 4d, 5p, 6s, 4f, 5d, 6p, 7s, 5f, 6d, 7p
	\item d-orbitaly se zaplňují až po zaplnění s-orbitalu s hlavním kvantovým číslem (n+1), např. 3d orbital se začne plnit až po 4s
	\item Zápis elektronové konfigurace: C: 1s$^2$ 2s$^2$ 2p$^2$; P: 1s$^2$ 2s$^2$ 2p$^6$ 3s$^2$ 3p$^3$
	\item Zkrácený zápis elektronové konfigurace: C: [He] 2s$^2$ 2p$^2$; P: [Ne] 3s$^2$ 3p$^3$
	\item U nepřechodných prvků (s a p blok PSP) je zaplňování orbitalů dáno jejich energetickým pořadím. Sb: [Kr] 4d$^{10}$ 5s$^2$ 5p$^3$
	\item U přechodných (d blok) a vnitřně přechodných (f blok) prvků nacházíme výjimky a nepravidelnosti v pořadí zaplňování orbitalů
	\end{itemize}
}

\frame{
	\frametitle{}
	\begin{itemize}
	\item \textbf{Změna pořadí energetických hladin}
	\end{itemize}

	\begin{tabular}{ll}
	K & [Ar] 4s$^1$ (3d$^0$ 4p$^0$) \\
	Ca & [Ar] 4s$^2$ (3d$^0$ 4p$^0$) \\
	\hline \hline
	Sc & [Ar] 3d$^1$ 4s$^2$ (4p$^0$) \\
	Ti & [Ar] 3d$^2$ 4s$^2$ (4p$^0$) \\
	\end{tabular}

	\begin{itemize}
	\item \textbf{Vyšší stabilita zpola zaplněných d-orbitalů}
	\item U prvků 6. a 11. skupiny dochází k přeskoku jednoho elektronu z~orbitalu s do orbitalu d, tím vzniká konfigurace se zpola nebo zcela zaplněným d-orbitalem.
	\item Cr: [Ar]  3d$^5$ 4s$^1$
	\item Cu: [Ar]  3d$^{10}$ 4s$^1$
	\item U f-prvků (lanthanoidy a aktinoidy) je elektronová konfigurace  (n–2)f$^{1-14}$(n–1)d$^{0-1}$ns$^2$
	\item Gd: [Xe] 4f$^7$ 5d$^1$ 6s$^2$
	\item U: [Rn] 5f$^3$ 6d$^1$ 7s$^2$
	\end{itemize}
}

\frame{
	\frametitle{}
	\begin{itemize}
	\item Při vzniku \emph{kationtů} se uvolňují elektrony z HOMO orbitalu (Highest Occupied Molecular Orbital - nejvyšší obsazený molekulový orbital).
	\item Při vzniku \emph{aniontů} elektrony vstupují do LUMO orbitalu (Lowest Unoccupied Molecular Orbital - nejnižší neobsazený molekulový orbital).
	\end{itemize}

	\begin{tabular}{|l|l|l|l|}
	\hline
	Na & [Ne] 3s$^1$ & Na$^+$ & [Ne] (3s$^0$) \\ \hline
	Ba & [Xe] 3s$^2$ & Ba$^{2+}$ & [Xe]  \\ \hline
	Fe & [Ar] 3d$^6$ 4s$^2$ & Fe$^{3+}$ & [Ar] 3d$^5$ \\ \hline
	Cu & [Ar] 3d$^{10}$ 4s$^1$ & Cu$^{2+}$ & [Ar] 3d$^9$ \\ \hline
	S & [Ne] 3s$^2$ 3p$^4$ & S$^{2-}$ & [Ne] 3s$^2$ 3p$^6 \equiv$ [Ar] \\ \hline
	Cl & [Ne] 3s$^2$ 3p$^5$ & Cl$^-$ & [Ne] 3s$^2$ 3p$^6 \equiv$ [Ar] \\ \hline
	\end{tabular}
}

\end{document}