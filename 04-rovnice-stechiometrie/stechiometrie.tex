\documentclass[hyperref=unicode]{beamer}

\usepackage[absolute,overlay]{textpos}
\usepackage{graphicx}
\usepackage{adjustbox}
\usepackage{mhchem}
\usepackage{wrapfig}
\usepackage{multirow}
\adjustboxset*{center}
\usepackage[utf8]{inputenc}
\usepackage{caption}

%dělení slov
\usepackage{ragged2e}
\let\raggedright=\RaggedRight
%konec dělení slov

\usepackage{fontspec}
\usepackage{unicode-math}

\usepackage{polyglossia}
\setdefaultlanguage{czech}

\def\uv#1{„#1“}

\mode<presentation>{\usetheme{Madrid}}
\DefineNamedColor{named}{pozadi}{RGB}{200,200,200}
\usecolortheme{crane}

\setbeamertemplate{footline}[frame number]

\addtobeamertemplate{frametitle}{
	\let\insertframetitle\insertsectionhead}{}
\addtobeamertemplate{frametitle}{
	\let\insertframesubtitle\insertsubsectionhead}{}

\makeatletter
\CheckCommand*\beamer@checkframetitle{\@ifnextchar\bgroup\beamer@inlineframetitle{}}
\renewcommand*\beamer@checkframetitle{\global\let\beamer@frametitle\relax\@ifnextchar\bgroup\beamer@inlineframetitle{}}
\makeatother
\setbeamercolor{section in toc}{fg=blue}
\setbeamertemplate{section in toc shaded}[default][100]

\usepackage{tikz}
\usetikzlibrary{positioning}
\usetikzlibrary{arrows}
\usetikzlibrary{shapes.multipart}

\title[Crisis]
{Chemické rovnice}

\subtitle{Vyčíslování chemických rovnic, stechiometrické výpočty}

\date{}

\begin{document}
\frame{\titlepage}

\section{Vyčíslování chemických rovnic}
\frame{
	\frametitle{}
	\begin{itemize}
	\item Na obou stranách rovnice musí být stejný počet atomů.
	\item V případě redoxních rovnic je nutné udělat i elektronovou bilanci.
	\item \ce{NaOH + H2SO4 -> Na2SO4 + H2O}
	\item \ce{2\ NaOH + H2SO4 -> Na2SO4 + 2\ H2O}
	\item Jednodušší neredoxní rovnice lze vyčíslit intuitivně.
	\item Složitější vyčíslujeme pomocí soustavy rovnic.
	\end{itemize}
}

\frame{
	\frametitle{}
\ce{a\ Na2WO4 + b\ SiO2 + c\ HCl -> d\ Na4[Si(W3O10)4] + e\ NaCl + f\ H2O}
\begin{tabular}{lr@{=}l}
Na & 2a & 4d + e\\
W & a & 12d\\
O & 4a + 2b & 40d + f\\
Si & b & d\\
H & c & 2f\\
Cl & c & e\\
\end{tabular}

Rovnice nejsou nezávislé, proto nemá soustava pouze jedno řešení. Volíme vždy takový výsledek, aby byly všechny koeficienty celočíselné a co nejmenší.
\\
\ce{12\ Na2WO4 + SiO2 + 20\ HCl -> Na4[Si(W3O10)4] + 20\ NaCl + 10\ H2O}
}

\section{Vyčíslování redoxních rovnic}
\frame{
	\frametitle{}
Kromě počtu atomů musíme brát do úvahy i počty vyměňovaných elektronů.\\
\ce{Sb2O3 + Br2 + KOH -> K3SbO4 + KBr + H2O}
\vspace{5mm}

Redukce: \ce{Br^0 -> Br^{-I}} - spotřebuje se jeden elektron\\
Oxidace: \ce{Sb^{III} -> Sb^{V}} - uvolní se dva elektrony\\
\vspace{3mm}
\begin{picture}(0,0)
\put(0,-1){Br: 1}
\put(60,-1){2}
\put(0,-32){Sb: 2}
\put(60,-32){1}
\put(25,0){\vector(1,-1){30}}
\put(25,-30){\vector(1,1){30}}
\end{picture}
\\
\vspace{5mm}
\raisebox{-8ex}{
\ce{Sb2O3 +2\ Br2 + 10\ KOH -> 2\ K3SbO4 + 4\ KBr + 5\ H2O}}
}

\section{Vyčíslování iontových rovnic}
\frame{
	\frametitle{}
Kromě počtu atomů a vyměňovaných elektronů musíme zajistit i rovnost nábojů na obou stranách rovnice.\\
\vspace{3mm}
\ce{Mn^{2+} + MnO4^- + H2O -> MnO2 + H^+}
\vspace{5mm}

Redukce: \ce{Mn^{VII} -> Mn^{IV}} - spotřebují se tři elektrony\\
Oxidace: \ce{Mn^{II} -> Mn^{IV}} - uvolní se dva elektrony\\
\vspace{3mm}
\begin{picture}(0,0)
\put(0,-1){Mn: 3}
\put(60,-1){2}
\put(0,-32){Mn: 2}
\put(60,-32){3}
\put(28,0){\vector(1,-1){30}}
\put(28,-30){\vector(1,1){30}}
\end{picture}
\\
\vspace{15mm}
\ce{3\ Mn^{2+} + 2\ MnO4^- + 2\ H2O -> 5\ MnO2 + 4\ H^+}
\\
Nábojová bilance: +4 = +4
}

\section{Stechiometrické výpočty}
\frame{
	\frametitle{}
\begin{itemize}
\item Je potřeba vycházet ze správně napsané a vyčíslené rovnice.
\item \ce{{Zn} + 2\ HCl -> ZnCl_2 + H_2}
\item $\frac{n_{Zn}}{\nu_{Zn}} = \frac{n_{HCl}}{\nu_{HCl}} = \frac{n_{ZnCl_2}}{\nu_{ZnCl_2}} = \frac{n_{H_2}}{\nu_{H_2}}$
\item $\frac{n_{Zn}}{1} = \frac{n_{HCl}}{2} = \frac{n_{ZnCl_2}}{1} = \frac{n_{H_2}}{1}$
\item \scalebox{1.5}{$\frac{n}{\nu} = konst$}
\item $n = \frac{m}{M} = \frac{\rho V}{M} = \frac{V}{V_m} = c.V$
\item Pokud je některá z reagujících látek v nadbytku, použijeme pro výpočet reaktant jehož poměr $\frac{n}{\nu}$ je nejmenší.
\item V případě plynných reaktantů nebo produktů, můžeme k výpočtu látkového množství využít molární objem (pozor na správný tlak a teplotu):
\item $n = \frac{V}{V_m} = \frac{V}{22,414}$
\item Hodnota V$_m$ je 22,414 dm$^3$.mol$^{-1}$ pro 0 °C a tlak 101,325~kPa.
\end{itemize}
}

\end{document}